\documentclass[a4paper,12pt,twosided]{article}
\usepackage{latexsym,amsthm,amsfonts,amsgen,amssymb,amscd,amsmath}
\usepackage[french]{babel}
\usepackage{rotating}
\usepackage[utf8]{inputenc}
\usepackage[T1]{fontenc}
%\usepackage{fullpage}
%\pagestyle{empty}
\usepackage[standard-baselineskips]{cmbright}
\usepackage{geometry} 
\geometry{a4paper, textwidth=7.4in, textheight=10in,
  marginparsep=9pt, marginparwidth=.6in,top=.8in}
\setlength\parindent{.6cm}
\setlength\itemsep{0in}
\setlength\parsep{.2cm}
\setlength\parskip{.2cm}
% \usepackage[inline]{enumitem}
\usepackage{enumerate}
\usepackage{mathrsfs}
\newcommand{\prob}{\mathbb P}
\newcommand{\esp}{\mathbb E}
\newcommand{\ee}{\mathrm e}
\newcommand{\ep}{\varepsilon}
\newcommand{\dd}{\mathrm{d}}
\newcommand{\ds}{\displaystyle}
\renewcommand{\mathcal}{\mathscr}
\newcommand{\Var}{\operatorname{Var}}
\newcommand{\var}{\operatorname{Var}}
\newcommand{\Loi}{\operatorname{Loi}}
\newcommand{\loi}{\operatorname{Loi}}
\newcommand{\st}{\,|\,}



\newtheorem{theoreme}{Théorème}
\theoremstyle{definition}
\newtheorem{exercice}{Exercice}
\newtheorem{probleme}{Problème}
\usepackage{sectsty} 
\usepackage{fancyhdr}

\parindent 0pt
\parskip 7pt


\fancyhead{}
\fancyhead[LO,RE]{Aix-Marseille Université}
\fancyhead[RO,LE]{M2 -- Mathématiques pour l'IA} % clear all header fields
\fancyfoot{} % clear all footer fields
\fancyfoot[LO,RE]{Projet noté}
\fancyfoot[RO,LE]{Compte-rendu à rendre}
\renewcommand{\headrulewidth}{0.4pt}
\renewcommand{\footrulewidth}{0.4pt}
\pagestyle{fancy}


\begin{document}

\pagestyle{fancy}
% -----

\begin{center}
  {\Large\textbf{Projet noté} : évaluation du risque par une méthode de Monte Carlo\\
    et tests par permutations}%\par
\end{center}
%\bigskip
%-----
\textbf{Une étoile $^\star$ marque les questions difficiles. Le compte-rendu est à rendre sur
  Ametice.}


L'objectif de ce TP est de regarder le test de comparaison de deux moyennes à partir de deux
échantillons indépendants.
Les données forment un vecteur $(x_1,\ldots, x_m, y_1,\ldots, y_n)$ de longueur $m+n$. Elles
concernent l'effet d'un traitement médical sur $m+n$ patients. Les $m=13$ premiers patients ont reçu
le médicament $A$, qui est le traitement de référence. Les $n=16$ autres patients ont reçu le
médicament $B$, qui est le nouveau traitement. L'objectif de l'étude est de savoir si ce médicament
$B$ est plus efficace que le traitement de référence.

Les données sont~:\\
$x_{1:m}^\text{obs} = (-3.06, -0.71, 11.99, 1.42, 1.84, 13.1, 4.19, -8.06, -3.96, -2.24, 9.61, 3.47, 3.77)$ et\\
$y_{1:n}^\text{obs} = (12.57, 7.44, 2.97, 10.35, 10.24, 9.89, 9.07, 8.23, 4.42, 2.9, 2.44, 0.49, 3.51, -3.05, 18.25, 12.29)$.

On modélise ces données $(x_1,\ldots, x_m, y_1,\ldots, y_n)$ par le vecteur de variables aléatoires
$(X_1,\ldots, X_m, Y_1,\ldots, Y_n)$. On suppose que ces $m+n$ variables sont indépendantes, que les
$X_i$ ont tous même loi, d'espérance $\mu_X$ et que les $Y_j$ ont tous même loi, d'espérance
$\mu_X+\Delta$. On s'intéresse au test 
\[
  H_0~: \Delta \le 0 \quad \textit{vs} \quad H_1~: \Delta>0.
\]


\section*{Partie 1~: Test de Student sous hypothèse gaussienne}

Dans cette partie, on suppose que
\begin{align}
  \label{eq:loi1}
  \loi(X_1) = \ldots = \loi(X_m) &= \mathscr N\Big(\mu_X, \sigma^2\Big), \quad \text{et} \
  \\
  \label{eq:loi2}
  \loi(Y_1) = \ldots = \loi(Y_n) &= \mathscr N\Big(\mu_X+\Delta, \sigma^2\Big).
\end{align}
Toutes les variables aléatoires du modèle suivent donc une loi gaussienne. En outre, on a supposé
que $\var(X_1)=\cdots = \var(X_m)=\var(Y_1)=\cdots=\var(Y_n)$.

On introduit
\begin{equation}\label{eq:T}
  T = \frac{\ds \bar Y - \bar X}{\ds \sqrt{S_p^2\left(\frac1m+\frac1n\right)}}
\end{equation}
où
\[
  \bar X = \frac1m\sum_{i=1}^m X_i, \quad \bar Y = \frac1n \sum_{j=1}^n Y_j,
  \quad
  S_X^2 = \frac1{m-1}\sum_{i=1}^m (X_i- \bar X)^2, \quad S_Y^2 = \frac1{n-1}\sum_{j=1}^n (Y_i- \bar Y)^2
\]
\[
  \text{et} \quad S_p^2 = \frac{1}{m+n-2}\Big( (m-1)S_X^2 + (n-1)S_y^2 \Big).
\]

On admettra les résultats suivant~:
\begin{itemize}
\item $\bar X$ varie autour de $\mu_X$,
\item $\bar Y$ varie autour de $\mu_Y$,
\item $S_p^2$ varie autour de $\sigma^2$,
\item et, sous l'hypothèse supplémentaire que (le vrai) $\Delta =0$, $T$ suit la loi de Student à
  $m+n-2$ degrés de liberté. 
\end{itemize}

L'objectif est d'abord de suivre la section 3 du plan de cours sur les tests statistiques pour
mettre en place le test de Student ici.

\begin{enumerate}[\bf {1.}1.]
\item Justifier du choix de $\Delta\le 0$ comme hypothèse nulle.
\item Justifier du choix de la statistique de test $T$.
\item Justifier que la zone de rejet soit de la forme $[c;+\infty[$, où $c$ est une constante à
  déterminer.
\item On note $\Phi_{m+n-2}$ la fonction de répartition de la loi de Student à $m+n-2$ degrés de
  liberté. Montrer que, si l'on fixe la taille du test à $\alpha$, il faut choisir
  $c=\Phi_{m+n-2}^{-1}(1-\alpha)$.
\item Écrire une fonction (sans boucle \texttt{for} explicite), nommée \verb+calculeT+ qui calcule
  la valeur observée de $T$, à partir de deux entrées~: le vecteur des $x_i$ et le vecteur des
  $y_i$. Que vaut la valeur observée $t^\text{obs}$ ici~?
\item En utilisant les fonctions de \verb+scipy.stats+, calculer $c$ dans notre cas si
  $\alpha = 0.05$. Quelle décision prend le test de Student~?
\item Montrer que la $p$-value est donnée ici par $p(X,Y) = 1 - \Phi_{m+n-2}(T)$. Quelle est la
  valeur observée de la $p$-value~?
\end{enumerate}

\section*{Partie 2~: Étude de la puissance du test de Student}

Dans cette partie, on conserve l'hypothèse de loi gaussienne des équations \eqref{eq:loi1} et
\eqref{eq:loi2} et on note $\theta=(\mu_X, \Delta, \sigma^2)$.
L'objectif est d'étudier la fonction puissance
\[
  \theta \mapsto \beta(\theta) = \prob_\theta\Big(T \ge \Phi_{m+n-2}^{-1}(1-\alpha)\Big).
\]
On s'intéresse aux valeurs de cette puissance lorsque
\begin{equation}
  \label{eq:intervalles}
  \Delta \in [0; 15],\quad \text{et} \quad  \sigma^2\in [20;40].
\end{equation}


\begin{enumerate}[\bf {2.}1.]
\item $^\star$ Montrer que la loi de $T$ ne dépend pas de $\mu_X$, mais uniquement de $\Delta$ et
  $\sigma^2$. Dans toute la suite, on fixera $\mu_X = 0.92$ si on a besoin d'une valeur numérique.
\item Écrire une fonction nommée \verb+simuleT+ qui prend en entrée les valeurs de $\Delta$ et
  $\sigma^2$ et qui fait les choses suivantes~:
  \begin{itemize}
  \item elle simule un vecteur $(X_1,\ldots, X_m)$ et un vecteur $(Y_1,\ldots, Y_n)$, de longueurs
    respectives $m=13$ et $n=17$~;
  \item elle calcule et renvoie la valeur de $T$ de l'équation \eqref{eq:T}, en utilisant la fonction \verb+calculeT+.
  \end{itemize}
\item Écrire une fonction \verb+puissT+ qui prend en entrée les valeurs de $\Delta$, $\sigma^2$, $\alpha$ et
  $N$ et qui fait les choses suivantes~:
  \begin{itemize}
  \item elle simule $N$ valeurs de $T$ indépendantes~;
  \item elle calcule et renvoie le nombre de fois où $T\ge \Phi_{m+n-2}^{-1}(1-\alpha)$
  \end{itemize}
\item En utilisant cette fonction pour $\sigma^2=20$ et $N=10^3$, approcher les valeurs de
  $\beta(\theta)$ pour les $16$ valeurs entières de $\Delta$ entre $0$ et $15$. Représenter
  graphiquement ces approximations en fonction de $\Delta$.
\item Même question pour $\sigma^2 = 30$.
\item Même question pour $\sigma^2=40$.
\item Que peut-on faire pour améliorer ces approximations~? Le faire, et constater le résultat.
\end{enumerate}

\section*{Partie 3 : calcul d'une $p$-value par une méthode de permutation Monte-Carlo}

On suppose maintenant que 
\begin{align}
  \label{eq:loi3}
  \loi(X_1) = \ldots = \loi(X_m) &\ \text{ de fonction de répartition }F(\cdot), \quad \text{et} \
  \\
  \label{eq:loi4}
  \loi(Y_1) = \ldots = \loi(Y_n) &\ \text{ de fonction de répartition }F(\cdot-\Delta).
\end{align}
Désormais, le paramètre inconnu est $\theta = (\Delta, F)$.

On note $(X_1^\ast,\ldots,X_m^\ast, Y_1^\ast, \ldots, Y_n^\ast)$ une permutation aléatoire du
vecteur $(X_1,\ldots, X_m, Y_1,\ldots, Y_n)$. Autrement dit, $X_1^\ast$ est tiré uniformément au hasard par le
vecteur $(X_1,\ldots, X_m, Y_1,\ldots, Y_n)$ de longueur $m+n$, puis $X_2^\ast$ est tiré
uniformément au hasard parmi les $m+n-1$ coordonnées restantes de $(X_1,\ldots, X_m, Y_1,\ldots,
Y_n)$,\ldots À la fin, $Y_n^\ast$ est la seule coordonnée de $(X_1,\ldots, X_m, Y_1,\ldots,
Y_n)$ qui n'a pas encore été tirée.
On pourra utiliser la fonction \verb+resample+ dans \verb+sklearn.utils+ pour faire cette
permutation aléatoire.

\begin{enumerate}[\bf {3.}1.]
\item Écrire une fonction \verb+approxP+, qui prend en entrée le vecteur des données $(x_1,\ldots,
  x_m, y_1, \ldots, y_n)$, $t^\text{obs}$ et $K$, et qui fait les
  choses suivantes~:
  \begin{itemize}
  \item elle calcule $K$ valeurs de $T$ sur $K$ échantillons permutés $(X_1^\ast,\ldots,X_m^\ast,
    Y_1^\ast, \ldots, Y_n^\ast)$ des données observées~;
  \item elle compte, parmis ces $K$ valeurs de $T$, le nombre de fois où $T>t^\text{obs}$~;
  \item elle renvoie la fréquence de $T>t^\text{obs}$ parmi ces $K$ simulations.
  \end{itemize}
\item $^\star$ En quelques lignes, expliquer pourquoi \verb+approxP+ permet d'approcher la $p$-value
  du test.
\item Rappeler comment on prend une décision en fonction de la valeur de la $p$-value et de la
  taille $\alpha$ du test.
\item Approcher la $p$-value du test sur les données qui nous intéresse par cette méthode avec
  $K=10^4$. Quelle est la décision que l'on prend ici.
\end{enumerate}

\section*{Partie 4~: Étude la puissance du test mis en place dans la partie 3}

L'objectif de cette partie est de comparer le test mis en place dans la partie 1, et celui mis en
place dans la partie 3, lorsque les hypothèses de la partie 1 sont satisfaites, c'est-à-dire lorsque
\eqref{eq:loi1} et \eqref{eq:loi2} sont satisfaites.
Comme dans la partie 2, on s'intéresse à des valeurs de $\Delta$ et $\sigma^2$ qui vérifient
\eqref{eq:intervalles} et pour $\mu_X=0.92$.
On note $\theta\mapsto \beta_3(\theta)$ la fonction puissance du test de la partie 3.

\begin{enumerate}[\bf {4.}1.]
\item En s'inspirant de la partie 2, écrire une fonction \verb+puiss3+ qui prend en entrée les
  valeurs de $\Delta$, $\sigma^2$, $\alpha$, $K$ et 
  $N$ et qui fait les choses suivantes~:
  \begin{itemize}
  \item elle simule $N$ jeux de données suivant \eqref{eq:loi1} et \eqref{eq:loi2}~;
  \item elle calcule, sur chacun de $N$ jeux de données, une approximation de la $p$-value à l'aide de
    \verb+approxP+ avec $K$ permutations indépendantes~;
  \item elle compte le nombre de fois où ces $N$ $p$-value sont inférieures à $\alpha$~;
  \item elle renvoie la fréquence où $p<\alpha$ parmi ces $N$ valeurs de $p$.
  \end{itemize}
\item En utilisant cette fonction pour $\sigma^2=20$, $N=10^3$ et $K=400$, approcher les valeurs de
  $\beta_3(\theta)$ pour les $16$ valeurs entières de $\Delta$ entre $0$ et $15$. Représenter
  graphiquement ces approximations en fonction de $\Delta$, ainsi que $\beta(\theta)$ tel que
  calculé dans la partie 2. On représentera les deux courbes dans deux couleurs différentes.
\item Même question pour $\sigma^2 = 30$.
\item Même question pour $\sigma^2=40$.
\item Quel est le test le plus puissant sous l'hypothèse gaussienne~?
\item Peut-on raisonnablement faire quelque chose ici pour améliorer l'approximation de $\beta_3(\theta)$~?
\end{enumerate}


\section*{Partie 5~: amélioration de l'approximation}

L'objectif de cette partie est d'étudier l'article de Boos et Zhang (2000, \textit{Journal of the
  American Statistical Association}). 

\begin{enumerate}[\bf {5.}1.]
\item Décrire en quelques lignes quel est l'objectif de l'article en question.
\item Quelle est l'idée de l'algorithme~?
\item $^\star$ Implémenter l'algorithme.
\item $^\star$ L'utiliser et conclure. 
\end{enumerate}

\end{document}
